\RequirePackage[l2tabu,orthodox]{nag}
\documentclass[10pt,twocolumn,a4paper]{article}

\usepackage[utf8]{inputenc}
\usepackage[T1]{fontenc}
\usepackage[english]{babel}
\usepackage{lmodern}
\usepackage{microtype}
\usepackage{parskip}
\usepackage{amsmath}
\usepackage{csquotes}

\usepackage[
	backend=biber
]{biblatex}
\addbibresource{gui.bib}

\usepackage[hidelinks]{hyperref}

\author{Markus Himmel}
\title{Windows GUI}

\newcommand{\bs}[1]{\textbf{\sffamily #1}}
\newcommand{\winver}[1]{$^{\text{\bs{#1}}}$}

\begin{document}
	\maketitle

	\begin{abstract}
		The Graphical User Interface is one of the most important components
		of a consumer-oriented operating system such as Microsoft Windows. This
		report examines the components of the Windows operating system that
		contribute to the graphical user interface, both from an application
		programmer's and a system architect's point of view.
	\end{abstract}

	\section{Introduction}
	\subsection{How to read this document}
		The Windows operating system has seen many revisions over its
		decades-long lifespan, and many of the features described in this
		report differ significantly from version to version. At the same time,
		documentation of the architecture of the Windows GUI is scarce at best.
		For this reason, it is not possible to present an entirely consistent
		exposition of the Windows GUI architecture of some fixed Windows
		version.  Instead, different sections of this report may be based on
		different versions of Windows. In order to aid the reader in keeping
		track of which version of Windows is being examined at any given point
		in time, sections and sometimes single paragraphs and sentences will be
		marked with the Windows versions they describe. Here is an example:

		GDI is not hardware-accelerated.\winver{V}

		Table~\ref{tbl:abbrev} lists all abbreviations used in this document
		and the Windows versions they refer to.

		\begin{table}[h]
			\begin{tabular}{r|l}
				Abbreviation & Windows Version \\
				\hline
				\bs{3} & Windows NT 3.51 \\
				\bs{4} & Windows NT 4.0 \\
				\bs{XP} & Windows XP \\
				\bs{V} & Windows Vista (\enquote{Longhorn}) \\
				\bs{7} & Windows 7 \\
			\end{tabular}
			\caption{Table of abbreviations for Windows versions used in this document}
			\label{tbl:abbrev}
		\end{table}

		\printbibliography
\end{document}
